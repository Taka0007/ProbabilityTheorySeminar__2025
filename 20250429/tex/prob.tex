\documentclass[uplatex,dvipdfmx,11pt]{jsarticle}

% ---------------- パッケージ ----------------
\usepackage{amsmath,amssymb,physics}
\usepackage{newtxtext,newtxmath}
\usepackage{xcolor}
\usepackage{fancyhdr}
\usepackage{titlesec}
\usepackage{tcolorbox}
\usepackage{geometry}
\geometry{margin=30mm}

% ---------------- デザイン変更 ----------------
\definecolor{mintgreen}{rgb}{0.7, 0.9, 0.7}
\definecolor{deepblue}{rgb}{0.6,0.75,0.6}
\definecolor{tea}{rgb}{0.6, 0.75, 0.6}  
\titleformat{\section}{\Large\bfseries\color{mintgreen}}{\thesection}{1em}{}
\renewcommand{\labelitemi}{$\\triangleright$}

% ヘッダー装飾
%\pagestyle{fancy}
%\fancyhf{}
%\fancyhead[L]{\textsc{ディラック関数と階段関数}}
%\fancyhead[R]{2025.04.29}
\pagestyle{empty}

% ---------------- 本文 ----------------
\begin{document}

\begin{center}
  {\LARGE \textbf{デルタ関数と階段関数}} \\[1ex]
  \rule{0.9\linewidth}{0.5pt}
\end{center}

以下のような関数を考える:

\begin{equation}
\theta(x) = 
\begin{cases}
  1 & (x > 0) \\\\
  0 & (x < 0)
\end{cases}
\end{equation}

この関数を \textbf{階段関数(Heaviside関数)} という。

\vspace{1em}


\begin{tcolorbox}[
  colframe=green!50!black,
  colback=green!5!white,
  title=(b) $\theta'(x)$ の積分値
]

階段関数の微分 $\theta'(x) = \dv{\theta}{x}$ が
\begin{equation} \tag{2}
  \int_{-\infty}^{\infty} \theta'(x) \, dx = 1
\end{equation}
となることを示せ。
\end{tcolorbox}

\vspace{1em}

\begin{tcolorbox}[
  colframe=green!50!black,
  colback=green!5!white,
  title=(b) $\delta(x)$ と関数への作用
]
$\varepsilon$ を正の定数としたとき、
\begin{equation} \tag{3}
  \int_{-\varepsilon}^{\varepsilon} f(x) \theta'(x) \, dx = f(0)
\end{equation}
となることを示せ。
\end{tcolorbox}

\vspace{1em}


\end{document}
